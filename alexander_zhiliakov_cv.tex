\documentclass[a4paper,12pt]{article}
\usepackage[utf8]{inputenc}
% \usepackage{nopageno}
\usepackage[left=1.5cm, right=1.5cm, top=1cm, bottom=1cm, includefoot, heightrounded]{geometry}
\usepackage[dvipsnames]{xcolor}
\definecolor{uofh}{RGB}{200, 16, 46}
\definecolor{56th}{RGB}{135, 82, 117}
\usepackage[colorlinks, urlcolor=uofh]{hyperref}
% centering
\usepackage{titlesec}
\titleformat*{\section}{\centering\large\bfseries}
\titleformat*{\subsection}{\large\bfseries}
%\titlespacing*{\subsection}{0pt}{15pt}{0pt}
% paragraph
\setlength\parindent{0pt}
% tabular
%\let\oldtabular\tabular
%\renewcommand{\tabular}{\def\arraystretch{1.5}\oldtabular}
\usepackage{longtable}
\usepackage{array} % https://tex.stackexchange.com/questions/338009/right-alignment-for-plength-box-in-tabular/338010#338010
\usepackage{etaremune}
\usepackage{titlesec}
\titlespacing*{\section}
{0pt}{3mm}{3mm}
\titlespacing*{\subsection}
{2mm}{3mm}{3mm}

\setlength{\LTpre}{0pt} % https://tex.stackexchange.com/questions/5683/how-to-remove-top-and-bottom-whitespace-of-longtable
\setlength{\LTpost}{3mm}

% https://tex.stackexchange.com/a/13897/135296 and
% https://latex.org/forum/viewtopic.php?p=104062&sid=25c3a1a121b0dea3f3c1642ebe9bf55f#p104062
\usepackage{fancyhdr}
\pagestyle{fancy}
\fancyhf{} % sets both header and footer to nothing
\renewcommand{\headrulewidth}{0pt}
\usepackage{lastpage}
\cfoot{\thepage\ of \pageref*{LastPage}}

\begin{document}

	\section*{CV of Alexander Zhiliakov}
	
	\begin{longtable}{ l >{\raggedright\arraybackslash}p{15cm} }
		Office:			& University of Houston, \href{http://www.uh.edu/nsm/math/}{Department of Mathematics}\\
		\phantom{Summer 2018} & 3551 Cullen Blvd (PGH Building), Houston, TX\,77204\vspace{2mm}\\
		Contacts:		& \href{https://telegram.me/fiftysixth}{@fiftysixth} (Telegram), \href{mailto:alex@math.uh.edu}{alex@math.uh.edu}\vspace{2mm}\\
		Homepage:		& \href{https://www.math.uh.edu/~alex}{math.uh.edu/${\sim}$alex}\vspace{2mm}\\
		Major skills:	& Scientific computing, applied mathematics, numerical analysis, (unfitted) finite element methods, preconditioners, computational fluid dynamics, computational haemodynamics, Unix\,/\,Windows, git, HPC (slurm, PAPI\,/\,TAU, OpenMP, Trilinos, Intel MKL), cmake, \texttt{C++\,11\,/\,17}, \texttt{FORTRAN}, \texttt{Wolfram Mathematica}\vspace{2mm}\\
		Languages:		& Russian (first), English (fluent)\vspace{3mm}\\
		\multicolumn{2}{p{17cm}}{I ask and answer questions on scientific computing on~\href{https://scicomp.stackexchange.com/users/21916/56th}{SciComp}. You can follow my research on~\href{https://www.researchgate.net/profile/Alexander_Zhiliakov}{ResearchGate}.}\\
	\end{longtable} 
	
	\subsection*{Experience}
	
	\begin{longtable}{ l >{\raggedright\arraybackslash}p{15cm} }
		2019 --
			& \textbf{Graduate Research Assistant}\vspace{1mm}\\
			& University of Houston, \href{http://www.uh.edu/nsm/math/}{Department of Mathematics}\vspace{1mm}\\
		Fall 2019
			& Goal: Modeling evolving fluidic interfaces with \href{https://www.math.uh.edu/~molshan/tracefinite.html}{trace FEM} discretization of (Navier--)Stokes equations posed on a surface \\
		Spring 2019
			& Goal: Implementing Taylor--Hood~($\mathbf{P}_2$--$P_1$) finite elements for trace FEM discretization of (Navier--)Stokes equations posed on a surface; Investigating inf-sup stability\vspace{1mm}\\
			& Contacts: Maxim Olshanskii \href{mailto:molshan@math.uh.edu}{molshan@math.uh.edu} (advisor), Arnold Reusken \href{mailto:reusken@igpm.rwth-aachen.de}{reusken@igpm.rwth-aachen.de}, Vladimir Yushutin \href{mailto:yushutin@math.uh.edu}{yushutin@math.uh.edu}\vspace{3mm}\\
		2018 --
			& \textbf{Graduate Research Assistant}\vspace{1mm}\\
			& Los Alamos National Laboratory, \href{http://www.lanl.gov/org/padste/adtsc/theoretical/applied-mathematics-plasma-physics}{Applied Mathematics and Plasma Physics Group}\vspace{1mm}\\
		Summer 2019
			& Goal: Generalizing analysis \& implementation of the \href{https://www.researchgate.net/publication/330912268_A_higher_order_approximate_static_condensation_method_for_multi-material_diffusion_problems}{higher order approximate static condensation (ASC)} algorithm to 3D\vspace{1mm}\\
		Summer 2018
			& Goal: Improving accuracy of the \href{https://www.researchgate.net/publication/318300724_Approximate_static_condensation_algorithm_for_solving_multi-material_diffusion_problems_on_meshes_non-aligned_with_material_interfaces}{ASC} algorithm for multi-material diffusion problems in the mixed form posed on general polygonal meshes\vspace{1mm}\\
			& Contacts: Daniil Svyatsky \href{mailto:dasvyat@lanl.gov}{dasvyat@lanl.gov} (mentor), Mikhail Shashkov \href{mailto:shashkov@lanl.gov}{shashkov@lanl.gov}\vspace{3mm}\\
		2017 --
			& \textbf{Teaching Assistant in Mathematics}\vspace{1mm}\\
			& University of Houston, \href{http://www.uh.edu/nsm/math/}{Department of Mathematics}\vspace{1mm}\\
			& Teaching, grading, and tutoring experience: Calculus I\,/\,II; Introduction to PDEs
	\end{longtable}

	\subsection*{Education}

	\begin{longtable}{ l >{\raggedright\arraybackslash}p{15cm} }
		2017 --
			& \textbf{PhD in Computational Science}\vspace{1mm}\\
		\phantom{Summer 2018} 
			& University of Houston, \href{http://www.uh.edu/nsm/math/}{Department of Mathematics}\vspace{1mm}\\
			& Contacts: Maxim Olshanskii \href{mailto:molshan@math.uh.edu}{molshan@math.uh.edu} (advisor), Ronald Hoppe \href{m
				ailto:rohop@math.uh.edu}{rohop@math.uh.edu}, 
			Annalisa Quaini \href{mailto:quaini@math.uh.edu}{quaini@math.uh.edu}, Yuri Kuznetsov \href{mailto:kuz@math.uh.edu}{kuz@math.uh.edu}\vspace{3mm}\\
%		2017 -- 2019
%			& \textbf{Master of Computer Science}\vspace{1mm}\\
%			& Novosibirsk State Technical University, \href{http://en.nstu.ru/faculties/faculty_of_applied_mathematics_and_computer_science/}{Faculty of Applied Mathematics and Computer Science}\vspace{1mm}\\
%			& Contacts: Marina Persova \href{mailto:persova@ami.nstu.ru}{persova@ami.nstu.ru} (advisor)\vspace{3mm}\\
		2013 -- 2017
			& \textbf{Bachelor of Computer Science}\\%, GPA: 3.7\,/\,4\vspace{1mm}\\
			& Novosibirsk State Technical University, \href{https://en.nstu.ru/education/faculty-of-applied-mathematics-and-computer-science/}{Faculty of Applied Mathematics and Computer Science}\vspace{1mm}\\
			& Contacts: Mikhail Balandin \href{mailto:balandin@corp.nstu.ru}{balandin@corp.nstu.ru} (advisor), Marina Persova \href{mailto:persova@ami.nstu.ru}{persova@ami.nstu.ru}\vspace{3mm}\\
		2010 -- 2012
			& \textbf{C++ Developer Qualification}\vspace{1mm}\\
			& \href{http://www.nadip.ru/}{Novosibirsk Design and Programming Academy}\\
	\end{longtable}

	\subsection*{Publications}
	
	\begin{etaremune}[topsep=0pt]
		\item \textit{\href{}{Extension of the approximate static condensation method for multi-material diffusion problems to 3D}}, A. Zhiliakov et al.\\ \href{}{LANL report}, September 2019 (in progress)
		\item \textit{\href{https://arxiv.org/abs/1909.02990}{Inf-sup stability of the trace $\mathbf P_2$--$P_1$ Taylor--Hood elements for surface PDEs}}, M. Olshanskii, A. Reusken, A. Zhiliakov\\ \href{}{Preprint}, September 2019 
		\item \textit{\href{https://www.researchgate.net/publication/333900759_A_higher_order_approximate_static_condensation_method_for_multi-material_diffusion_problems}{A higher order approximate static condensation method for multi-material diffusion problems}}, A. Zhiliakov, D. Svyatsky, M. Olshanskii, E. Kikinzon, M. Shashkov\\ \href{https://www.sciencedirect.com/science/article/pii/S0021999119304528}{Journal of Computational Physics}, June 2019 
%		\item \textit{\href{https://www.researchgate.net/publication/330912268_A_higher_order_approximate_static_condensation_method_for_multi-material_diffusion_problems}{A Higher Order Approximate Static Condensation Method for Multi-Material Diffusion Problems}}, A. Zhiliakov, D. Svyatsky, M. Olshanskii, E. Kikinzon, M. Shashkov\\ Journal of Computational Physics (preprint), 2019 
		\item \textit{\href{https://www.researchgate.net/publication/329327346_A_higher_order_approximate_static_condensation_method_for_multi-material_diffusion_problems}{A higher order approximate static condensation method for multi-material diffusion problems}}, A. Zhiliakov, D. Svyatsky, M. Olshanskii, E. Kikinzon, M. Shashkov\\ \href{https://permalink.lanl.gov/object/tr?what=info:lanl-repo/lareport/LA-UR-18-31169}{LANL report LA-UR-18-31169}, November 2018 
		\item \textit{\href{https://www.researchgate.net/publication/318039077_Postroenie_i_realizacia_fiziceskih_pereobuslavlivatelej_dla_zadac_vycislitelnoj_gemodinamiki}{Construction and implementation of physics-based preconditioners for problems of computational haemodynamics}}, A. Zhiliakov\\ \href{https://elibrary.nstu.ru/source?id=61216}{BCS Thesis}, June 2017   
%		\item \href{https://github.com/CATSPDEs/}{CATS'\,PDEs} research papers:
%		\begin{itemize}
%			\item \textit{Non-linear Iteration with Anderson Mixing for an Inverse Problem of Magnetostatics}, joint with M.\,A. Olshanskii, in progress 
%			
%			\item \textit{\href{https://github.com/CATSPDEs/CATSPDEs/blob/master/sln/FEM for Magnetostatic Poisson Problem/LaTeX/FEM for Magnetostatic Poisson Problem.pdf}{Finite Element Method for Magnetostatic Poisson Problem}}, A. Zhiliakov, March 2017
%			
%			\item \textit{\href{https://github.com/CATSPDEs/papers/tree/master/FEM (and FDM) for Elliptic (Parabolic, Hyperbolic) 2D Problems/main.pdf}{Finite Element and Finite Difference Methods for Hyperbolic and Parabolic 2D IBVPs}}, A. Zhiliakov, October 2016 
%		\end{itemize}
	\end{etaremune}

	\subsection*{Conferences\,/\,Talks}
	
	\begin{etaremune}[topsep=0pt]
		\item \href{https://www.siam.org/Conferences/CM/Main/gs19}{SIAM Conference on Mathematical \& Computational Issues in the Geosciences}, Houston, TX, 12 March 2019\\
		      Poster: \textit{\href{https://www.researchgate.net/publication/331674777_Generalized_approximate_static_condensation_method_for_a_heterogeneous_multi-material_diffusion_problem}{Generalized approximate static condensation method for a heterogeneous multi-material diffusion problem}}
		\item \href{https://www.siam.org/Conferences/CM/Main/cse19}{SIAM Conference on Computational Science and Engineering}, Spokane, WA, 27 Feb 2019\\
		      Talk: \textit{A higher order approximate static condensation method for multi-material diffusion problems}
		\item \href{https://www.siam.org/Conferences/CM/Main/txla18}{SIAM Louisiana-Texas Section Conference}, Baton Rouge, LA, 7 October 2018\\
		      Talk: \textit{Generalized approximate static condensation method for a heterogeneous multi-material diffusion problem}
		\item \href{https://ami.nstu.ru/o-fakultete/news/1699/}{FAMCS NSTU Student Scientific Conference} on Numerical Modeling in Physics, Novosibirsk, Russia, March 2017\\
		      Talk: \textit{\href{https://www.researchgate.net/publication/316884102_Mnogosetocnye_metody_resenia_ellipticeskih_zadac}{Multigrid techniques for solving elliptic problems}}
	\end{etaremune}

	\subsection*{Software}
	
	\begin{longtable}{ l >{\raggedright\arraybackslash}p{15cm} }
		\multicolumn{2}{p{17cm}}{This is the list of software I contributed to and\,/\,or used in my research.}\vspace{3mm}\\
		\href{https://www.igpm.rwth-aachen.de/DROPS/}{DROPS}
			& CFD tool for simulating two-phase flows (\texttt{C++})\vspace{1mm}\\
			& Used and augmented during Spring and Fall 2019 UH research assistantship for surface PDEs simulations\vspace{3mm}\\
		\href{https://github.com/amanzi}{AMANZI}
			& Multi-process HPC simulator (\texttt{C++})\vspace{1mm}\\
			& Used and augmented during Summer 2018 and Summer 2019 LANL research assistantship for implementation of the ASC method\vspace{3mm}\\
		\href{https://github.com/laristra/tangram}{TANGRAM}
			& Framework for interface reconstruction in computational physics applications (\texttt{C++})\vspace{1mm}\\
			& Used (with minor contributions) during Summer 2019 LANL research assistantship for implementation of the moment-of-fluid interface reconstruction in 3D\vspace{3mm}\\
		\href{https://github.com/CATSPDEs/}{CATS'\,PDEs}
			& Modern OOP toolkit for efficient implementation of typical FEM~routines (\texttt{C++}, \texttt{FORTRAN}, and \texttt{Mathematica})\vspace{1mm}\\
			& Developed during 2016\,--\,2017 as a part of the BCS project
	\end{longtable}
	
	\vfill
	
	\begin{flushright}
		\small
		\color{Gray}{\today\\ \url{https://www.math.uh.edu/~alex/alexander_zhiliakov_cv.pdf}}
	\end{flushright}
		
\end{document} 